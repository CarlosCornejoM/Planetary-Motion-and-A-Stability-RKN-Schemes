%% ================================
%% Packages =======================
\documentclass[letter, 11pt]{article}
%% ================================
%% Packages =======================
\usepackage[utf8]{inputenc}      %%
\usepackage[T1]{fontenc}         %%
\usepackage{lmodern}             %%
\usepackage[spanish]{babel}      %%
\decimalpoint                    %%
\usepackage{fullpage}            %%
\usepackage{fancyhdr}            %%
\usepackage{graphicx}            %%
\usepackage{amsmath}             %%
\usepackage{color}               %%
\usepackage{mdframed}            %%
\usepackage[colorlinks]{hyperref}%%
%% ================================
%% ================================

%% ================================
%% Page size/borders config =======
\setlength{\oddsidemargin}{0in}  %%
\setlength{\evensidemargin}{0in} %%
\setlength{\marginparwidth}{0in} %%
\setlength{\marginparsep}{0in}   %%
\setlength{\voffset}{-0.5in}     %%
\setlength{\hoffset}{0in}        %%
\setlength{\topmargin}{0in}      %%
\setlength{\headheight}{54pt}    %%
\setlength{\headsep}{1em}        %%
\setlength{\textheight}{8.5in}   %%
\setlength{\footskip}{0.5in}     %%
%% ================================
%% ================================

%% =============================================================
%% Headers setup, environments, colors, etc.
%%
%% Header ------------------------------------------------------
\fancypagestyle{firstpage}
{
  \fancyhf{}
  \lhead{\includegraphics[height=4.5em]{LogoDFI.jpg}}
  \rhead{FI3104-1 \semestre\\
         Métodos Numéricos para la Ciencia e Ingeniería\\
         Prof.: \profesor}
  \fancyfoot[C]{\thepage}
}

\pagestyle{plain}
\fancyhf{}
\fancyfoot[C]{\thepage}
%% -------------------------------------------------------------
%% Environments -------------------------------------------------
\newmdenv[
  linecolor=gray,
  fontcolor=gray,
  linewidth=0.2em,
  topline=false,
  bottomline=false,
  rightline=false,
  skipabove=\topsep
  skipbelow=\topsep,
]{ayuda}
%% -------------------------------------------------------------
%% Colors ------------------------------------------------------
\definecolor{gray}{rgb}{0.5, 0.5, 0.5}
%% -------------------------------------------------------------
%% Aliases ------------------------------------------------------
\newcommand{\scipy}{\texttt{scipy}}
%% -------------------------------------------------------------
%% =============================================================
%% =============================================================================
%% CONFIGURACION DEL DOCUMENTO =================================================
%% Llenar con la información pertinente al curso y la tarea
%%
\newcommand{\tareanro}{5}
\newcommand{\fechaentrega}{17/10/2022 22:59 hrs}
\newcommand{\semestre}{2022B}
\newcommand{\profesor}{Valentino González}
%% =============================================================================
%% =============================================================================


\begin{document}
\thispagestyle{firstpage}

\begin{center}
  {\uppercase{\LARGE \bf Tarea \tareanro}}\\
  Fecha de entrega: \fechaentrega
\end{center}


%% =============================================================================
%% ENUNCIADO ===================================================================
\noindent{\large \bf Problema}

El problema de interacción gravitacional de 3 cuerpos es en general caótico pero
existen varias soluciones \emph{interesantes}, algunas de ellas incluso
estables. A pesar de ser un problema antiguo, varias de las soluciones
\emph{interesantes} han sido descubiertas recientemente.

En este problema, Ud. debe integrar numéricamente las trayectorias de 3 cuerpos
que interactúan gravitacionalmente. Dadas las condiciones iniciales que
consideraremos, las trayectorias están restringidas al plano $x-y$. Las
condiciones iniciales a considerar son las siguientes:

\vspace{0.5em}
Masa 1:
$\vec{r_0} = [-0.97000436, 0.24308753], \vec{v_0}= [0.466203685, 0.43236573]$ 

Masa 2:
$\vec{r_0} = [0.97000436, -0.24308753], \vec{v_0}= [0.466203685, 0.43236573]$ 

Masa 3:
$\vec{r_0} = [0, 0], \vec{v_0}= [-0.93240737, -0.86473146] $ 

\vspace{0.5em}
Además, consideraremos: $G = m_1 = m_2 = m_3 = 1$ (esto fija las unidades de las
condiciones iniciales indicadas más arriba).

{\bf Esta tarea debe ser resuelta utilizando OOP.} El archivo
\texttt{codigo/planetario.py} contiene el esqueleto de la clase
\texttt{TresCuerpos}. Ud. debe implementar los métodos de esta clase para
resolver el problema. Los \texttt{docstrings} explican en qué debe consistir
cada método. Ud. tiene libertad de mejorar los \texttt{docstrings}, agregar
atributos y métodos a la clase según le parezca conveniente.

En particular, debe:

\begin{enumerate}

  \item Implementar pasos temporales utilizando ya sea el método de Verlet o el
    de Beeman. Utilice el esqueleto del método llamado
    \texttt{verlet\_o\_beeman} y cámbiele el nombre según corresponda. Debe
    implementar el método.

  \item Implementar pasos temporales utilizando el método de Runge-Kutta (al
    menos orden 4). No es necesario que lo implemente para esta tarea. Como
    sugerencia, si bien puede usar una librería (por ejemplo \texttt{scipy}),
    probablemente sería más sencillo usar el código que ya implementó en una
    tare anterior.

  \item Por simplicidad, para esta tarea no usaremos paso adaptativo, puede
    definir un paso pequeño y usar el mismo paso para toda la integración. Debe
    decir qué pruebas hizo para decidir que el paso utilizado era adecuado.

  \item Debe calcular las trayectorias de las 3 partículas con los 2 métodos
    escogidos y compararlas. Como sugerencia, cree 2 objetos de la clase
    \texttt{TresCuerpos} e integre uno con Verlet (o Beeman) y el otro con RK
    para compararlos.

  \item Debe estudiar la energía total del sistema y ver cómo evoluciona en los
    2 métodos de integración. Comente sobre las similitudes o diferencias en su
    informe.

\end{enumerate}


\begin{ayuda}
  \small
  \noindent {\bf Visualización.}

  La clase \texttt{TresCuerpos} contiene el método
  \texttt{show\_animated\_orbit} que muestra una animación de las órbitas de los
  3 planetas. Para poder usarla, debe tener \texttt{matplotlib} instalado pero
  {\bf NO ES NECESARIO USAR ESTE MÉTODO}, es sólo una adición que le puede
  parecer interesante. También pueden usar el código como inspiración para
  escribir otros métodos de visualización de las órbitas o para generar gráficos
  estáticos.

\end{ayuda}

%% FIN ENUNCIADO ===============================================================
%% =============================================================================

\vspace{2em}
\noindent\textbf{Información e Instrucciones Importantes.}
\begin{itemize}

\item Repartición de puntaje:

  \subitem Código: 5\% (aprobar PEP8) + 5\% (uso de git) + 50\% resolución del
  problema, implementación de los algoritmos necesarios, considerando además:
  uso efectivo de nombres, modularidad, calidad y utilidad de los
  \texttt{docstrings}.
  
  \subitem Informe 40\%: Calidad del reporte considerando: análisis de las
  soluciones, comparación de los métodos en términos de las órbitas y de las
  energías del sistema. Claridad del lenguaje, calidad de las figuras y/o
  tablas. Como referencia, probablemente 3 páginas son suficientes para esta
  tarea.

\item \textbf{REVISE SU ORTOGRAFÍA.}

\end{itemize}
%%%%%%%%%%%%%%%%%%%%%%%%%%%%%%%%%%%%%%%%%%%%%%%%%%%%%%%%%%%%%%%%%%%%%%%%%%%%%%%
%%%%%%%%%%%%%%%%%%%%%%%%%%%%%%%%%%%%%%%%%%%%%%%%%%%%%%%%%%%%%%%%%%%%%%%%%%%%%%%
\end{document}
